%%%%%%%%%%%%%%%%%%%%%%%%%%%%%%%%%%%%%%%%%
% Awesome Cover Letter
% XeLaTeX Template
% Version 1.1 (9/1/2016)
%
% This template has been downloaded from:
% http://www.LaTeXTemplates.com
%
% Original authors:
% Claud D. Park (posquit0.bj@gmail.com)
% Lars Richter (mail@ayeks.de)
% With modifications by:
% Vel (vel@latextemplates.com)
%
% License:
% CC BY-NC-SA 3.0 (http://creativecommons.org/licenses/by-nc-sa/3.0/)
%
% Important note:
% This template must be compiled with XeLaTeX, the below lines will ensure this
%!TEX TS-program = xelatex
%!TEX encoding = UTF-8 Unicode
%
%%%%%%%%%%%%%%%%%%%%%%%%%%%%%%%%%%%%%%%%%

%----------------------------------------------------------------------------------------
%	PACKAGES AND OTHER DOCUMENT CONFIGURATIONS
%----------------------------------------------------------------------------------------

\documentclass[11pt, a4paper]{awesome-cv} % A4 paper size by default, use 'letterpaper' for US letter

\geometry{left=2cm, top=1.5cm, right=2cm, bottom=2cm, footskip=.5cm} % Configure page margins with geometry
 
\fontdir[fonts/] % Specify the location of the included fonts

% Color for highlights
\colorlet{awesome}{awesome-nephritis} % Default colors include: awesome-emerald, awesome-skyblue, awesome-red, awesome-pink, awesome-orange, awesome-nephritis, awesome-concrete, awesome-darknight
%\definecolor{awesome}{HTML}{CA63A8} % Uncomment if you would like to specify your own color

% Colors for text - uncomment and modify
%\definecolor{darktext}{HTML}{414141}
%\definecolor{text}{HTML}{414141}
%\definecolor{graytext}{HTML}{414141}
%\definecolor{lighttext}{HTML}{414141}

\renewcommand{\acvHeaderSocialSep}{\quad\textbar\quad} % If you would like to change the social information separator from a pipe (|) to something else

%----------------------------------------------------------------------------------------
%	PERSONAL INFORMATION
%	Comment any of the lines below if they are not required
%----------------------------------------------------------------------------------------

\name{Christoph}{Biesinger}
\address{Rathausstrasse 50, 86343 Königsbrunn}
\mobile{(+49) 0152/25777314}

\email{mail@christophbiesinger.de}
%\homepage{christophbiesinger.de}
\github{chrb}
\linkedin{christophbiesinger}
\xing{Christoph\_Biesinger}
%\skype{mail\_87492}
%\stackoverflow{SOid}{SOname}
%\twitter{@twit}

\position{Engineer - Access and Security} % Your expertise/fields
%\quote{``Make the change that you want to see in the world."} % A quote or statement

%----------------------------------------------------------------------------------------
%	RECIPIENT/POSITION/LETTER INFORMATION
%	All of the below lines must be filled out
%----------------------------------------------------------------------------------------

\recipient{Scandio GmbH}{Frölichstraße 14\\86150 Augsburg} % The company being applied to

\letterdate{\today} % The date on the letter, default is the date of compilation

\lettertitle{Bewerbung als Cloud Systems Engineer in Augsburg} % The title of the letter

\letteropening{Sehr geehrter Herr Köberle,} % How the letter is opened

\letterclosing{Mit freundlichen Grüßen,} % How the letter is closed

%\letterenclosure[Anhang]{Lebenslauf} % Any enclosures with the letter

\makecvfooter{\today}{Christoph Biesinger Anschreiben}{} % Specify the letter footer with 3 arguments: (<left>, <center>, <right>), leave any of these blank if they are not needed
  
%----------------------------------------------------------------------------------------

\begin{document}

\makecvheader % Print the header

\makelettertitle % Print the title

%----------------------------------------------------------------------------------------
%	LETTER CONTENT
%----------------------------------------------------------------------------------------

\begin{cvletter}

%------------------------------------------------

%\lettersection{About Me}

vielen Dank für das freundliche und informative Telefonat.

Bislang habe ich Erfahrungen vor allem in der Anwendungsbereitstellung mit dem Reverse Proxy NetScaler sammeln können. Als Engineer und Consultant eines Beratungsunternehmens bin ich den täglichen Umgang mit Kunden und den Anforderungen von Projekten und des Betriebs von IT Systemen vertraut. Aufbauend auf diesen Kenntnissen möchte ich die Scandio und ihre Kunden unterstützen.

Meine Kenntnisse will ich technisch künftig auf die gesamte Infrastruktur von Anwendungen ausbauen. Dafür experimentiere ich in meinem Homelab und in AWS um schrittweise Technologien und Herausforderungen im Cloud Native Umfeld zu lernen.

Am meisten begeistert mich dabei Docker Container und deren Orchestrierung mit Kubernetes. Die Administration und vor allem die geänderte und deklarative Arbeitsweise mit Infrastructure as Code sind interessant. 
Ein besonderes Augenmerk lege ich natürlich auf die Auswirkungen der Anwendungsbereitstellung. Das nun von Natur aus dynamische Verhalten sorgt für spannende Aufgaben und neue Möglichkeiten durch die Verwendung eines Service Mesh.


%------------------------------------------------

%\lettersection{Why Me?}

Meine Gehaltsvorstellungen liegen zwischen 55.000 und 60.000 Euro pro Jahr. Mein bevorzugter Einsatzort ist Augsburg. 
Bei weiteren Fragen stehe ich Ihnen natürlich per Mail oder telefonisch zur Verfügung. Ich freue mich auf das persönliche Gespräch.


%------------------------------------------------

\end{cvletter}

%----------------------------------------------------------------------------------------

\makeletterclosing % Print the signature and enclosures

\end{document}
